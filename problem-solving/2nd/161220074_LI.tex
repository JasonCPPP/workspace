\documentclass[a4paper,UTF8]{article}
\usepackage{ctex}
\usepackage[margin=1.25in]{geometry}
\usepackage{color}
\usepackage{graphicx}
\usepackage{amssymb}
\usepackage{amsmath}
\usepackage{amsthm}
\usepackage{booktabs}
\usepackage{caption}
\usepackage{fancyhdr}
\usepackage{extramarks}
\usepackage{amsfonts}
\usepackage{tikz}
\usetikzlibrary{arrows}
\usepackage{algorithm}
\usepackage{algorithmicx}
\usepackage{algpseudocode}
\usepackage{listings}

%\usepackage[thmmarks, amsmath, thref]{ntheorem}
\theoremstyle{definition}
\newtheorem*{solution}{Solution}
\newtheorem*{prove}{Proof}
\usepackage{multirow}

%
% Basic Document Settings
%

\topmargin=-0.45in
\evensidemargin=0in
\oddsidemargin=0in
\textwidth=6.5in
\textheight=9.0in
\headsep=0.25in

\linespread{1.1}

\pagestyle{fancy}
\lhead{\hmwkTitle}
\chead{\hmwkClass\ (\hmwkClassInstructor\ \hmwkClassTime)}
\rhead{\hmwkAuthorName}
\lfoot{\lastxmark}
\cfoot{\thepage}

\renewcommand\headrulewidth{0.4pt}
\renewcommand\footrulewidth{0.4pt}

%
% Homework Details
%   - Title
%   - Due date
%   - Class
%   - Section/Time
%   - Instructor
%   - Author
%

\newcommand{\hmwkTitle}{Assignment \ \#2}
\newcommand{\hmwkDueDate}{September 11, 2017}
\newcommand{\hmwkClass}{Problem Solving}
\newcommand{\hmwkClassTime}{}
\newcommand{\hmwkClassInstructor}{Professor Chen}
\newcommand{\hmwkAuthorName}{李志琦 161220074}

%--

%--
\begin{document}

\section*{Chapter 25 }

\subsection*{25.1-4}

  要证明算法中矩阵的乘法满足结合率,要满足:

  $$(L_iL_j)L_k = L_i(L_jL_k)$$
  Answer:\\
  \begin{eqnarray*}
  (L_iL_j)L_k=M_1\\
  L_i(L_jL_k)=M_2\\
  M_1[a][b] & = & min_{1\leq m \leq n}(L_iL_j)[a][m]+L_k[m][b]\\
            & = & min_{1\leq m \leq n}min_{1\leq k \leq n}(L_i[a][k]+L_j[k][m]+L_k[m][b])\\
            & = & min_{1\leq k \leq n}(L_i[a][k]+min_{1\leq m \leq n}(L_j[k][m]+L_k[m][b]))\\
            & = & min_{1\leq k \leq n}(L_jL_k)[k][b]+L_i[a][k]\\
            & = & M_2[a][b]\\
           M_1[a][b]=M_2[a][b]
  \end{eqnarray*}
所以满足结合率
\subsection*{25.1-5}
我们很容易在all pairs 的最终结果矩阵中,通过某一行的数据就可以构造出来单元最短路树,
所以如果只求一颗单元最短路树,我们要的就是矩阵的那一行而已。
所以我们只用一个vector v表示所有点到源的距离,首先初始化,跟矩阵的初始化一样。
然后 $v_{i+1}=v_iW$  当算到$v_{n-1}$就得到了正确的答案,所以乘(n-2)次就行了。\\
与bellman ford 算法的相似性:

相乘就相当于bellman ford的relax


\subsection*{25.1-6}
一个n*n的矩阵,我们需要算出没每个位置的父节点,所以我们需要一个二重循环,

每$n^2$次循环的每一次循环中,我们知道$L_{ij}=L{ik}+w(k,j$,所以枚举n次k,我们就能找到答案。


\subsection*{25.1-9}
\begin{algorithm}[h]
  \caption{FASTER-ALL-PAIRS-SHORTEST-PATHS($W$)}
  \begin{algorithmic}[1]
   \item n=W.rows
   \item l[1] = W
   \item m = 1
  \While {m<n-1}
    \State let l[2m] be a new n*n marticx
    \State l[2m] = EXTENDED-SHORTEST-PATHS(l[m],l[m])
    \State m=2m;
  \EndWhile

  \item l[n] =EXTENDED-SHORTEST-PATHS(l[m],l[m])

  \If {l[m]\ !=\ l[n]}
        \State has negative-weight circle
  \EndIf
  \end{algorithmic}
\end{algorithm}

\subsection*{25.1-10}
如果有负圈,那么当当一个circle之后 $W[i][i]<0$;
所以在slow all pairs 算法中每extend一次我们就检查一下是否有W[i][j]<0
如果有那么当前的循环的i就是圈的长度
\subsection*{25.2-2}
矩阵初始化的时候也是有边就赋值为1(自身也是1),没边赋值为0,
算min的时候改为$l[i][j]|(l[i][k]\&w(i,k))$
最后矩阵中为1的就表示有边


\subsection*{25.2-4}

我们将原来的$O(n^3)$和现在的$O(n^2)$
对比,现在的算法只用了一个矩阵,但是看关系式子
$$d_{ij}=min(d_{ij},d_{ik})+d_{kj}$$
原来的算法的式
$$d_{ij}^m=min(d_{ik}^{m-1},d_{ik}^{m-1}+d_{kj}^{m-1})$$
因为在更新新的矩阵$d_{ij}$之前$d_{ij}$的值其实就是$d_{ij}^{m-1}$\\
而$d_[ik]+d_{kj}$至少不必$d_{ik}^{m-1}+d_{kj}^{m-1}$小,因为我们要的是更小,所以\\
$d_[ij]$得到的结果至少不必的另一个算法的$d_{ij}^m$小,所以最后依然能得到正确的结果。\\

\subsection*{25.2-6}
只要矩阵中对角线上有小于零的值,就有副回路,应为,如果有负回路,那么一圈下来,点到自身的距离就会小于0.

 \subsection*{24.2-8}

 首先构造一个n*n的矩阵M,有边为1,没边为0,
$|v|$个点每一点v,做一个循环,然后dfs(v,v)
\begin{algorithm}[h]

  \caption{dfs($v_1$,$v$)}
  \begin{algorithmic}[1]
   \item M[$v_1$][$v$]=1
   \For { each u in G.adj[$v_1$]}
        \If {!vis[u]}  \# let v is be a n-array,to index if vertice u is visited
          \State dfs(u,$v$)
        \EndIf
  \EndFor
  \end{algorithmic}
\end{algorithm}
每一个dfs(v,v)是$O(E)$,所以总代价是$O(EV)$

\subsection*{25.3-2}
加上这个点,我们肯定能判断出来是否有负回路,但是如果我们只是从原来图上任意选择一个,如果图不是联通的,那么有可能检测不到负圈。\\
如果没有负回路,那么得到$h(v)=\delta (s,v)$,能够得b到符合条件的h(v)函数
\subsection*{25.3-3}
相等,因为h(v)=0对于任意一点v\\
因为 $h(v)=\delta (s,v)$,但是因为所有的边都是正值,所以$\delta (s,v)=0$
\subsection*{25-2}
\subsubsection*{a}
最小堆和最大堆的效率是一样的,所以参考$6-2$, $Insert,extract-min,decrease_key$的代价分别为:
$O(\log_{d}{n}),O(d\log_{d}{n}),O(\log_{d}{n})$\\
if $d=\theta(n^{\alpha})$ 三个代价分别为$O(\frac{1}{\alpha}),O(\frac{n^{\alpha}}{\alpha}),O(\frac{1}{\alpha})$\\
$fibonacci\_heap$ 的三个操作代价以此为$O(1),O(\log{n}),O(1)$
\subsubsection*{b}
另$d=n^{\varepsilon }$,因为总n次$ECTRACT-MIN$和最多E次$descease-key$操作 ,所哟
$O(nd\log_{d}{n}+E(1/\varepsilon))$
.\\
$O(\frac{n^{1+ \varepsilon }+E}{\varepsilon})$

$O(n^{1+\varepsilon})$

$O(E)$

\subsection*{c}
每一个节点都用b中的算法
\subsection*{d}
转换weight function,增加一个节点,然后计算新的权重,和书中的方法一样。

\end{document}
